% !TeX root = ../main.tex

\chapter{插图和表格}

\section{插图}

插图中的术语、符号、单位等应与正文表 述所用一致。
专用名词、代号、 单位等可采用外文表示,坐标轴题名、词组、描述性的词语均须采用中文。

图题置于图的下方,图名末尾不可加标点。
若有英文图题,另起一行置于中文图题下方。
推荐使用 \pkg{bicaption} 宏包,例如图~\ref{fig:example}。

\begin{figure}[h]
  \centering
  \includegraphics[width=.4\linewidth]{example-image.pdf}
  \bicaption{图题置于图的下方}{If there is an English caption, place it below the Chinese caption}
  \label{fig:example}
  \figurenote{若有图注,图注置于图题下方。
    若有英文图题,则只需在英文图题下方给出英文图注。
    多个图注则须顺序编号,注序左缩进 2 字,与注文之间空一字符,续行悬挂缩进左对齐,两端对齐。
    注文的字数较少且是短语时,末尾不可加标点,
    多个图注可以在同一行通过自由选取字符空格将各个图注间隔开来;
    注文的字数较多或者甚至需要用句子说明时,该图注可以独立成行。
  }
\end{figure}

如果插图中的图是由两个以上的分图组成时,则各分图要求分别以 (a)、(b)、(c)……作为分图序,
与分图名一起构成分图题。
将分图题按图注处理,分图题的格式与图注同。
推荐使用 \pkg{subcaption} 宏包,例如图~\ref{fig:sub-figures}。

\begin{figure}[h]
  \centering
  \subcaptionbox{分图示例 A\label{fig:sub-figure-a}}
    {\includegraphics[width=.4\linewidth]{example-image-a.pdf}}
  \subcaptionbox{分图示例 B\label{fig:sub-figure-b}}
    {\includegraphics[width=.4\linewidth]{example-image-b.pdf}}
  \caption{插图由两个以上的分图组成的示例}
  \label{fig:sub-figs}
\end{figure}


\section{表格}

表题置于表的上方。若有英文表题,另起一行置于中文表题下方。

\begin{table}[h]
  \centering
  \bicaption{表号和表题在表的正上方}{If there is an English caption, place it below the Chinese caption}
  \label{tab:example}
  \begin{tabular}{cl}
    \toprule
    类型   & 描述                                       \\
    \midrule
    挂线表 & 挂线表也称系统表、组织表,用于表现系统结构 \\
    无线表 & 无线表一般用于设备配置单、技术参数列表等   \\
    卡线表 & 卡线表有完全表,不完全表和三线表三种       \\
    \bottomrule
  \end{tabular}
\end{table}

若表格有附注,则表注置于表的下方。
推荐使用 \pkg{threeparttable} 宏包,例如表~\ref{tab:table-note}。

\begin{table}[h]
  \centering
  \begin{threeparttable}
    \caption{带附注的表格示例}
    \label{tab:table-note}
    \begin{tabular}{cl}
      \toprule
      类型   & 描述                                       \\
      \midrule
      挂线表\tnote{1} & 挂线表也称系统表、组织表,用于表现系统结构 \\
      无线表\tnote{2} & 无线表一般用于设备配置单、技术参数列表等   \\
      卡线表          & 卡线表有完全表,不完全表和三线表三种       \\
      \bottomrule
    \end{tabular}
    \begin{tablenotes}[online]
      \item[1] 多个表注则须顺序编号,注序左缩进 2 字,与注文之间空一字符,
        续行悬挂缩进左对齐,两端对齐。
      \item[2] 注文若是短语,末尾不可加标点。
    \end{tablenotes}
  \end{threeparttable}
\end{table}

当表格较大而不能在一页内打印时,可以“续表”的形式另页打印,格式同前,
只需在另页的表序前加“续”字即可,例如“续表 2.1 实验方式”。
续表均应重复给出横表头。
推荐使用 \pkg{longtable} 宏包,例如表~\ref{tab:long-table}。

\begin{longtable}{cccc}
    \caption{跨页长表格的表题}
    \label{tab:long-table} \\
    \toprule
    表头 1 & 表头 2 & 表头 3 & 表头 4 \\
    \midrule
  \endfirsthead
    \caption*{续表~\thetable\quad 跨页长表格的表题} \\
    \toprule
    表头 1 & 表头 2 & 表头 3 & 表头 4 \\
    \midrule
  \endhead
    \bottomrule
  \endfoot
  Row 1  & & & \\
  Row 2  & & & \\
  Row 3  & & & \\
  Row 4  & & & \\
  Row 5  & & & \\
  Row 6  & & & \\
  Row 7  & & & \\
  Row 8  & & & \\
  Row 9  & & & \\
  Row 10 & & & \\
\end{longtable}


\section{算法环境}

算法环境可以使用 \pkg{algorithms} 或者 \pkg{algorithm2e} 宏包。

\renewcommand{\algorithmicrequire}{\textbf{输入:}}
\renewcommand{\algorithmicensure}{\textbf{输出:}}

\begin{algorithm}
  \caption{Calculate $y = x^n$}
  \label{alg:example}
  \begin{algorithmic}
    \REQUIRE $n \geq 0$
    \ENSURE $y = x^n$

    \STATE $y \leftarrow 1$
    \STATE $X \leftarrow x$
    \STATE $N \leftarrow n$

    \WHILE{$N \neq 0$}
      \IF{$N$ is even}
        \STATE $X \leftarrow X \times X$
        \STATE $N \leftarrow N / 2$
      \ELSE[$N$ is odd]
        \STATE $y \leftarrow y \times X$
        \STATE $N \leftarrow N - 1$
      \ENDIF
    \ENDWHILE
  \end{algorithmic}
\end{algorithm}
