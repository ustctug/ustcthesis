\chapter{数学}

\section{定理、引理和证明}

\begin{definition}
    If the integral of function $f$ is measurable and non-negative, we define
    its (extended) \textbf{Lebesgue integral} by
    \begin{equation}
        \int f = \sup_g \int g,
    \end{equation}
    where the supremum is taken over all measurable functions $g$ such that
    $0 \leq g \leq f$, and where $g$ is bounded and supported on a set of
    finite measure.
\end{definition}

\begin{example}
    Simple examples of functions on $\mathbb{R}^d$ that are integrable
    (or non-integrable) are given by
    \begin{equation}
        f_a(x) =
        \begin{cases}
            |x|^{-a} & \text{if } |x| \leq 1,\\
            0 & \text{if } x > 1.
        \end{cases}
    \end{equation}
    \begin{equation}
        F_a(x) = \frac{1}{1 + |x|^a}, \qquad \text{all } x \in \mathbb{R}^d.
    \end{equation}
    Then $f_a$ is integrable exactly when $a < d$, while $F_a$ is integrable
    exactly when $a > d$.
\end{example}

\begin{lemma}[Fatou]
    Suppose $\{f_n\}$ is a sequence of measurable functions with $f_n \geq 0$.
    If $\lim_{n \to \infty} f_n(x) = f(x)$ for a.e. $x$, then
    \begin{equation}
        \int f \leq \liminf_{n \to \infty} \int f_n.
    \end{equation}
\end{lemma}

\begin{remark}
    We do not exclude the cases $\int f = \infty$,
    or $\liminf_{n \to \infty} f_n = \infty$.
\end{remark}

\begin{corollary}
    Suppose $f$ is a non-negative measurable function, and $\{f_n\}$ a sequence
    of non-negative measurable functions with
    $f_n(x) \leq f(x)$ and $f_n(x) \to f(x)$ for almost every $x$. Then
    \begin{equation}
        \lim_{n \to \infty} \int f_n = \int f.
    \end{equation}
\end{corollary}

\begin{proposition}
    Suppose $f$ is integrable on $\mathbb{R}^d$. Then for every $\epsilon > 0$:
    \begin{enumerate}
        \renewcommand{\theenumi}{\roman{enumi}}
        \item There exists a set of finite measure $B$ (a ball, for example) such that
        \begin{equation}
            \int_{B^c} |f| < \epsilon.
        \end{equation}
        \item There is a $\delta > 0$ such that
        \begin{equation}
            \int_E |f|  < \epsilon \qquad \text{whenever } m(E) < \delta.
        \end{equation}
    \end{enumerate}
\end{proposition}

\begin{theorem}
    Suppose $\{f_n\}$ is a sequence of measurable functions such that
    $f_n(x) \to f(x)$ a.e. $x$, as $n$ tends to infinity.
    If $|f_n(x)| \leq g(x)$, where $g$ is integrable, then
    \begin{equation}
        \int |f_n - f| \to 0 \qquad \text{as } n \to \infty,
    \end{equation}
    and consequently
    \begin{equation}
        \int f_n \to \int f \qquad \text{as } n \to \infty.
    \end{equation}
\end{theorem}

\begin{proof}
    Trivial.
\end{proof}

\section{自定义}
\newtheorem{legend}{传说}
\begin{legend}
    龙生龙,凤生凤,华罗庚的弟子会打洞。
\end{legend}

\chapter{Citation}

Test\cite{mittelbach04},
Test\cite{lamport94},
Test\cite{knuth86a,lamport94,mittelbach04},
Test\cite{刘海洋2013}.
