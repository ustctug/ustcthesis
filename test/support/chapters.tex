\frontmatter
\begin{innovations}
  在此须说明本学位论文的创新性,确保符合学术博士学位论文的创新要求。
\end{innovations}

\ustcsetup{
  keywords  = {学位论文, 摘要, 关键词},
  keywords* = {Dissertation, Abstract, Keywords},
}

\begin{abstract}
  摘要内容。
\end{abstract}

\begin{abstract*}
  Abstract contents.
\end{abstract*}

\tableofcontents

\begin{notation}
  Notations.
\end{notation}

\mainmatter
\chapter{简介}
Chapter text. \par
Another paragraph.
\section{一级节标题}
Section text.
\subsection{二级节标题}
Subsection text.
\subsubsection{三级节标题}
Subsubsection text.
\paragraph{四级节标题}
Paragraph text.
\subparagraph{五级节标题}
Subparagraph text.

\chapter{内容要素}

\section{有关图、表和表达式}

图、表和表达式按章连续编号,用两个阿拉伯数字表示,前一数字为章的序号,
后一数字为本章内图、表或表达式的顺序号。
两数字间用半角小数点“.”连接。
例如“图~\ref{fig:example}”、“表~\ref{tab:example}”、“式\eqref{eq:example}”等等。

\subsection{插图}

插图一般由图、图题、图注构成。

\begin{figure}[h]
  \centering
  \includegraphics[width=0.3\textwidth]{figures/ustc-badge.pdf}
  \caption{图题置于图的下方}
  \label{fig:example}
  \figurenote{若有图注,图注置于图题下方。
    注文的字数较少且是短语时,末尾不可加标点,多个图注可以在同一行通过自由选取字符空格将各个图注间隔开来;
    注文的字数较多或者甚至需要用句子说明时,该图注可以独立成行。
  }
\end{figure}

不宜将多个插图连续排版,插图与文字论述段落应穿插排版。

\subsection{表格}

表格一般由表、表题、表注构成。

\begin{table}[h]
  \centering
  \caption{表题置于表的上方}
  \label{tab:example}
  \begin{tabular}{cl}
    \toprule
    类型   & 描述                                       \\
    \midrule
    挂线表 & 挂线表也称系统表、组织表,用于表现系统结构 \\
    无线表 & 无线表一般用于设备配置单、技术参数列表等   \\
    卡线表 & 卡线表有完全表,不完全表和三线表三种       \\
    \bottomrule
  \end{tabular}
\end{table}

不宜将多个表格连续排版,表格与文字论述段落应穿插排版。

\clearpage

\subsection{表达式}

表达式主要是指分章连续编有序号的数字表达式。
\begin{equation}
  w_1 = u_{11} - u_{12} u_{21}
  \label{eq:example}
\end{equation}
表达式的行距根据需要采用,段前6磅、段后6磅。

\chapter{引用}
\ustcsetup{cite-style=super}
\cite{knuth86a}

\begin{thebibliography}{1}
\bibitem[Knuth(1986)]{knuth86a}
KNUTH~D~E.
\newblock Computers and typesetting: volume~A\quad The
  {\TeX}book\allowbreak[M].
\newblock Reading, MA, USA: Addison-Wesley, 1986.
\end{thebibliography}

\appendix
\chapter{附录章节}
附录内容。

\backmatter
\begin{acknowledgements}
  致谢内容。
\end{acknowledgements}

\begin{achievements}

\begin{theachievements}[已发表论文]
  \item xx. 中国科学技术大学研究生学位论文撰写规范. 中国科学技术大学学报,2024,1.
\end{theachievements}

\begin{theachievements}[发明专利]
  \item xx. 一种温热外敷药制备方案. 授权号:CN123456789B,2023-08-12.
\end{theachievements}

\end{achievements}
