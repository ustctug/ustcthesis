\frontmatter
\ustcsetup{
  keywords  = {学位论文, 摘要, 关键词},
  keywords* = {Dissertation, Abstract, Keywords},
}

\begin{abstract}
  摘要内容。
\end{abstract}

\begin{abstract*}
  Abstract contents.
\end{abstract*}

\tableofcontents

\begin{notation}
  Notations.
\end{notation}

\mainmatter
\chapter{简介}
Chapter text. \par
Another paragraph.
\section{一级节标题}
Section text.
\subsection{二级节标题}
Subsection text.
\subsubsection{三级节标题}
Subsubsection text.
\paragraph{四级节标题}
Paragraph text.
\subparagraph{五级节标题}
Subparagraph text.

测‘引号’和“双引号”。

Test “double” quotaion ‘marks’.

\chapter{插图和表格}

插图一般由图、图题、图注构成。插图中的术语、符号、单位等应与正文表述所用一致。
图序与图名合称图题,例如:“图2.1  2010年至2024年西藏地区年平均气温分布”。
“图2.1”是图序,是“第二章第1个图”的序号,依次类推。

\begin{figure}[h]
  \centering
  \includegraphics[width=0.3\textwidth]{figures/ustc-badge.pdf}
  \caption{图题置于图的下方}
  \label{fig:badge}
  \note{若有图注,图注置于图题下方。注文的字数较少且是短语时,末尾不可加标点,多个图注可以在同一行通过自由选取字符空格将各个图注间隔开来;注文的字数较多或者甚至需要用句子说明时,该图注可以独立成行。}
\end{figure}

不宜将多个插图连续排版,插图与文字论述段落应穿插排版,插图前后与文字段落之间须留有10磅空行。
插图如果不得不与其它插图或表格连续排版,则它们之间须留有20磅空行。插图不允许跨页。

\begin{table}[h]
  \centering
  \caption{表题置于表的上方}
  \label{tab:exampletable}
  \begin{tabular}{cl}
    \toprule
    类型   & 描述                                       \\
    \midrule
    挂线表 & 挂线表也称系统表、组织表,用于表现系统结构 \\
    无线表 & 无线表一般用于设备配置单、技术参数列表等   \\
    卡线表 & 卡线表有完全表,不完全表和三线表三种       \\
    \bottomrule
  \end{tabular}
\end{table}

不宜将多个表格连续排版,表格与文字论述段落应穿插排版,表格前后与文字段落之间须留有10磅空行。
表格如果不得不与其它表格或插图连续排版,则它们之间须留有20磅空行。

\chapter{引用}
\ustcsetup{cite-style=super}
\cite{knuth86a}

\backmatter
\begin{thebibliography}{1}
\bibitem[Knuth(1986)]{knuth86a}
KNUTH~D~E.
\newblock Computers and typesetting: volume~A\quad The
  {\TeX}book\allowbreak[M].
\newblock Reading, MA, USA: Addison-Wesley, 1986.
\end{thebibliography}

\appendix
\chapter{附录章节}
附录内容。

\begin{acknowledgements}
  致谢内容。
\end{acknowledgements}

\begin{achievements}

\begin{theachievements}[已发表论文]
  \item xx. 中国科学技术大学研究生学位论文撰写规范. 中国科学技术大学学报,2024,1.
\end{theachievements}

\begin{theachievements}[发明专利]
  \item xx. 一种温热外敷药制备方案. 授权号:CN123456789B,2023-08-12.
\end{theachievements}

\end{achievements}
